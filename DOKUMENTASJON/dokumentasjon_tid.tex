\section{Tid}

For å ikkje måtte oppdatere alle objekta i neuralnettet kvar tidsiterasjon, har eg laga konseptet: flytende tid.

Grunnen til å ha tid i det heile tatt er:
\begin{itemize}
	\item at en del aspekter for objekta er tidsvarinte (funksjon av tid). Bl.a. 'refraction time' og lekkasje.
	\item for å få rett kausalitet i det simulerte sytemet. --at  $b_2$ skjer ETTER $b_1$, men samtidig med $a$ om $a$ og $b_2$ skal skje samtidig. Får (simulert) samtidighet.
\end{itemize}

Dette med simulert samtidighet er også eit poeng. Vidare skal vi også få inn 'refraction time' for neurona. %XXX

For å få til dette med tid, uten at alle objekta må skjekkes kva tidsiterasjon, vil eg bruke konseptet flytende tid. For å beskrive dette, må eg først beskrive systemets 'scheduler'.

\subsection{'Scheduler'}
For å få til kausalitet (at $a_2$ skal skje etter $a_1$, men ikkje etter $b$) har eg laga en 'scheduler'. 

Se for deg ei lenka liste som inneholder alle jobbane, pNesteJobbArbeidsKoe. En (alltid nøyaktig en) av jobbane er en spesiell tybe jobb, kalla tid.doTask();. I tillegg har eg en (globalt tilgjengelig) variabel kalla ulTidsiterasjoner.

Alle jobbane har muligheten for å skape nye jobber, som i såfall legges på på slutten av pNesteJobbArbeidsKoe.

Når \emph{tid.doTask()} kalles, vil denne gjøre de aktiviteter som skal gjøres i alle tidsiterasjonane og iterere tid ( ulTidsiterasjoner++ ).
I tillegg skal denne legge på en \emph{[this]} peiker på slutten av arbeidslista \emph{pNesteJobbArbeidsKoe}.

Dette vil skape en lenka liste variant av at det er to arbeidskøer som altererer, med tidsiterator kvar gang ei av de er tom og vi skifter til andre arbeidslista.



\subsection{Klassene}
Alle som skal schedules skal arve fra class tidInterface. Alle klassene som skal ha timing (causalitet og tidsvariante funksjoner) skal arve fra tidInterface. 
Dette er ei interface-klasse med \emph{virtal void doTask() =0;}

I tillegg til alle klassene som har elementer som er funksjoner av tid (for eksempel lekkasjen i 'leaky integrator'), skal en spesialklasse \emph{class tid} arver fra tidInterface.

\subsubsection{tid::pNesteJobbArbeidsKoe}
\emph{tid::pNesteJobbArbeidsKoe} inneholder alle jobber som skal gjøres. Dersom denne er tom, slutter programmet å gjøre ting (i tillegg til at tid slutter --også \emph{tid} er eit element i arbeidslista).

Dersom en jobb skaper nye jobber, legges desse til på slutten av lista.

Ei klasse er heilt spesiell når det kommer til tid: i tillegg til å inneholde \emph{static list pNesteJobbArbeidsKoe}, fungerer tidsSkilleElement som eit skilleElement mellom to tidsiterasjoner. 
Når \emph{tid.doTask()} kalles, skal alt som har med tidsprogresjonen gjøres. I tillegg tenker eg å putte funksjoner som går som bare en funksjon av tid inn her (alt som skal skje i kvar tidsiterasjon)..

Men det viktigaste \emph{tid.doTask()} gjør, er å iterere tid.
\emph{tid} har dermed ansvaret for å skille mellom to tidsiterasjoner.

\subsection{implementasjon}
Som sagt: alle klasser som har ei oppgave som skal gjøres (i den simulerte tida) puttes inn i \emph{pNesteJobbArbeidsKoe}. 

Selve schedulinga skjer i 
\begin{equation}
	\text{void* schedulerFunskjon(void*)}
\end{equation}
Grunnen til formatet på funksjonen er muligheten for å starte tråder (konvensjoner i pthread--library).

Denne funksjonen kaller pNesteJobbArbeidsKoe.front()->doTask(). Når pNesteJobbArbeidsKoe er av type [\emph{class tid}] vil tid itereres.

I tillegg skal alt anna som skal skje kvar einaste iterasjon plasseres her. Eksempel kan være at nye synapser blir lagt i ei kø, og schedulerFunskjon går gjennom denne køa kvar gang den kalles. 
Events kan kanskje også plasseres her?

Eller eg kan gjøre desse andre tinga med egne tråder?
