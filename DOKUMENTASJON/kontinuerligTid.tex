% Notater (etter) H 2011

\section{Kontinuerlig tid}
\label{secKontinuerligTid}
For flyt-baserte noder, kan vi oppnå nær-kontinuerlig tid for depol. forløpet.
Vi kan beregne når neste fyring vil være, og seie at denne skal skje nøyaktig da. 
Ved fyring vil eit nytt 'time window' initialiseres ved å sette $v_0 = 0$ og tid $t_0=t^*$.
Vi vil dermed får eksakt depol. forløp også for starten av neste time window.

	\subsection{'time window'}
	Ved nytt input (endra $\kappa$ eller ved fyring) vil eit nytt 'time window' bli initialisert.
	For å få til kontinuerlig tid (float--accuracy for tidspunkt), trenger vi float-tidspunkt for initialiseringen av det nye 'time window'.

	Første vi må gjør er å la $t_0$ være gitt av en \emph{double} istaden for en \emph{unsigned long}.
	Vi vil dermed prøve å definere \emph{double dStartOfTimeWindow} istadenfor \emph{unsigned long ulStartOfTimeWindow}.
	
		\subsubsection{fyring}
		Dette er det letteste tilfellet. Ved fyring har vi allerede beregna eit eksakt fyringstidspunkt.
		Fyringstidspunktet eg også definert som start--tid for neste 'time window', og  vi trenger bare lagre dette tidspunktet.

		Første vi må gjøre er å definere \emph{double dEstimatedTaskTime} istadenfor \emph{unsigned long ulEstimatedTaskTime}.
		I \emph{K\_auron::doTask()} lagrer vi auronets \emph{dEstimatedTaskTime} til \emph{dStartOfTimeWindow}.
		TRUR DETTE VIL GI OSS PERFEKT OPPLADING AV NEURONET!
	
% TODO 1) Lagre estimert tidspunkt i en double-variabel istadenfor en unsigned long: double dEstimatedTaskTime 			ikkje lenger ulEstimatedTaskTime.. 														TODO (1)
% TODO 2) Lagre tidspunkt for start av 'time window' i en double istadenfor unsigned long: 	dStartOfTimeWindow 			ikkje lenger ulStartOfTimewindow.. 														TODO (2)
%		 	2.1) Trenger å endre på getCalculateDepol() slik at denne kan kalkulere depol for eit float-tidspunkt. Argument!
% 				Har innført getCalculateDepol(double). KOMMENTERT UT! Går for løyringa under: innfører variabel 		dNextStartOfTimeWindow
% TODO  	2.2) Trenger å ha inn tidspunkt til doCalculations(): kanskje bedre å innføre variable dNextStartOfTimeWindow? (gjelder også den over..)
% 			2.3) Har delt opp doCalculation() i [beregn ny kappa] og [estimer periode]. Estimering av periode skjer no i funksjonen protected inline void K_auron::estimatePeriod();
		\subsubsection{input til neuronet}
			Når neuronet får endret input, vil tidsspunkt for endring være viktig for å beholde eksakt løysing.
			(I starten tenker eg å gjøre dette lett, og sette $t_0$ til enten starten av iterasjonen eller midten av iteraasjonen. (trur dette gir samme resultat i lange løp). 
				Etterkvart må eg utvide dette til å være gitt av input. Input kan for eksempel ha ved en "phase", som gir når den er gyldig. 
				Dersom vi lar postsyn. $t_0$ være gitt av overføringens $t_0$ skalert med inflytelsen til overføringen, kan vi få til dette målet. Jobb meir med dette seinare).

		Foreløpig sjekker vi en sensor node(som sampler eit input, og ikkje har noko neuralt input).
		Vi kan da definere at sampling skjer ved starten av kvar tidsiterasjon.
		Dermed kan vi definere at 'time window' (float-tidspunkt) er det samme som tidsiterasjonens nummer (unsigned long-tidspunkt - sample nummer).

% TODO 3) Lagre starten av 'time window', initialisert av en endret senset variabel, $\kappa$, ved å bare typekonvertere tidsiterasjonen: 			dStartOfTimeWindow = (double)ulTime; //(sett den til "no") 	TODO (3)

	\subsection{Normalisert tid: mulighet for å endre tidsoppløysinga}
	For å få samme transiente forløp for simuleringer med ulik tidsoppløsning, må vi ikkje bare referere til tid som tidsstegnummer. 
	Vi må da skalere på basis av lengden på computational time step, $\Delta t$.
	Dette kan vi gjøre ved å skalere med $\alpha$ (bare lekkasjen [OG OGSÅ MENGDEN INPUT] som er avhengig av lengden på eit tidssteg.
	Finn ut om eg kan løse dette bare ved å endre $\alpha$!
	%- Skaler både den nedskrevene tida (i log-filene) og tida som sendes inn i sensor-funksjon. Det viktigasete er det siste, men dei henger sammen..

	%TODO Plasser all kode som har med å skriv til logg i mdl.funk. writeDepolToLog() -også for SANN: TODO
	%TODO 	Så innfør skalering i denne funksjonen. (da blir alt slikt samla på samme sted i koden (KANN og SANN writeDepolToLog() kan ligge rett ved kvarandre!
