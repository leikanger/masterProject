\documentclass[a4paper,9 pt]{article}

\usepackage[utf8]{inputenc}
%\usepackage[T1]{fontenc}
%\usepackage{babel}
\usepackage{graphicx}

\usepackage[font=small,format=plain,labelfont=bf,up,textfont=it,up]{caption}

\usepackage{amsmath}
\usepackage{listings}

\author{Per R. Leikanger}
                             
\title{Dokumentasjon --Kva eg tenker mens eg koder det.}
\date{\today}     

\begin{document}   

\maketitle

\section{Introduction}
I denne fila skal eg dokumentere kva eg tenker, etterkvart som eg koder. Bra for seinare, når eg skal skrive rapport om det...

%Anngående globale variabler: Skriv i rapporten kvifor eg har brukt referanse-returnerande funksjoner i staden for globale variabler. Referer [Stroustrup, 2008] (3. edition, kap 9) Side 207, der han skriv at det er bra å minimalisere bruk av globale variabler. Referanse-returnerande funk er godt alternativ.
%Sjå s 217.

Kanskje eg skal bruke funksjonsobjekt (for for\_each() ) til bruk på arbeidsliste? Trur det i såfall blir mest for å brife med kunnskap / skrive i rapporten..

Kan også brife med file--streams. Utskrift til matlab-fil for plotting. Dette er også relevant.


%%%%%%%% NYTT. Skriv om i masteren. %%%%%%%%
\include{v2012}
\section{Notater: H.2011}
	\subsection{Input}
	For en SN er input gitt av inn-flyten gitt av frekvens og størrelse av overføringene (i kvart tidssteg).
	Lekkasjen blir simulert eksplisitt, og er ikkje en del av ligningen.

	For KN derimot, har vi fra utledinga av $v(t)$, at $\kappa = \frac{I}{\alpha}$. 
	Dersom vi seier at sensinga for en sensornode påvirker aktivitetsnivået ($\kappa$) direkte, må flytte over $\alpha$ og får ligninga for synaptisk input i en sensornode:
\begin{equation}
	I = \kappa \alpha
\end{equation}
	% Sjå ALPA123@neuroElement.cpp for SN-variant, og

	\subsection{Refraction Period}
	Når man programmerer vil feil oppstå. 
	'Refraction period' blir utregna heilt forskjellig for de to ANN modellane: SANN simulerer det og KANN bruker ligningene.
	For å minimalisere antall potensielle feilkilder har eg tatt refraction period ut av simuleringene mine. 
%	(Eg har tatt bort testen for om bBlockInput_refractionTime for SANN:  Sjå id: asdf21344@neuroElement.cpp)
% 	(Eg har også tatt vekk refraction time ved å ikkje lenger legge til N på estimert periode, der N er refraction time. Før var det +1. No er det kommentert ut.)

	\subsection{ANNET}
	Har endret mekanismen som runder til rett tidssteg: Før lå denne på beregninga av dLastCalculatedPeriod. Nå ligger det på den plassen der dette blir omgjort til ulEstimatedTaskTime.	

	Har tatt vekk recalkulateKappa for K_sensor_auron: Vi antar at sensor auronet ikkje kan motta annen neural input! sjå: asdf41412@neuroElement.cpp
	%(NEI: så på K_sensor_auron::recalculateKappa(): Det einaste denne gjorde var 'return 0;̈́')

	\subsection{FEIL med KANN}
	- starter oppladnina av depol. samme iterasjon som den fyrer. Dette er feil i forhold til SANN: Skal starte iterasjonene etter (?)
	(KVEN ER RETT?)
	Siden vi ikkje har refraction period, skal den starte ved det 'time instant' som neuronet fyrer. Dette kan eg implementere ved å lagre fyringstidspunkt (og dermed 'start of time window' som en float). DO IT!

	HOVEDPROBLEMET er at KANN later opp for fort. KVIFOR? Eksakt utregna første fyring for konst. kappa=1.1*Tau er tid:599.4  
	SANN:600 	KANN: 591
	Eg har dermed funnet at SANN fyrer for seint (som forutsett). Problemet er at KANN ikkje funker heilt. KVIFOR?
	K_sensor_auron starta med K=sensed value. Oppdaterte første tidssteg til (sensedValue-0)*ALPHA. Dette gjorde at Kappa ble for høg resten av tida. DRITT!
	Innførte at K_sensor_auron::K_sensor_auron() også lagra dLastSensedValue til å være Kappa, ved init av K_sensor_auron. No funker det for statisk sensorfunk!

\section{Legg til task i presentTimeIteration}
Har lagt til funksjonen time\_class::addTaskInPresentTimeIteration(K_auron* pK_auronArg) som kan legge til eit K\_auron til nåværende tidsiter.
Dette for å unngå at depol. går langt over terskel.
(Før brute-force'a eg eit AP dersom depol var over terskel. Trur dette blir unødvendig med dette nye tillegget!)

FETT, trur det funker nå. Men er ikkje sikker fordi eg vil skrive artikkel istadenfor å utvikle meir.
Pluss at eg har eksamen om ei veke :(



\section{PLAN}
	- innfør float-tid for KANN. (sjå \ref{secKontinuerligTid}).

	- Skaler både den nedskrevene tida (i log-filene) og tida som sendes inn i sensor-funksjon. Det viktigasete er det siste, men dei henger sammen..

\section{ROT!}
Det er noko rot med alpha! Eg ganger med alpha for s\_dendrite::newInputSignal(). OG ingenting med K\_auron::changeKappa\_derived().

Er dette rett, eller skal det være motsatt?

MEN DET GIR VEL SAMME RESULTATET?

% Notater (etter) H 2011

\section{Kontinuerlig tid}
\label{secKontinuerligTid}
For flyt-baserte noder, kan vi oppnå nær-kontinuerlig tid for depol. forløpet.
Vi kan beregne når neste fyring vil være, og seie at denne skal skje nøyaktig da. 
Ved fyring vil eit nytt 'time window' initialiseres ved å sette $v_0 = 0$ og tid $t_0=t^*$.
Vi vil dermed får eksakt depol. forløp også for starten av neste time window.

	\subsection{'time window'}
	Ved nytt input (endra $\kappa$ eller ved fyring) vil eit nytt 'time window' bli initialisert.
	For å få til kontinuerlig tid (float--accuracy for tidspunkt), trenger vi float-tidspunkt for initialiseringen av det nye 'time window'.

	Første vi må gjør er å la $t_0$ være gitt av en \emph{double} istaden for en \emph{unsigned long}.
	Vi vil dermed prøve å definere \emph{double dStartOfTimeWindow} istadenfor \emph{unsigned long ulStartOfTimeWindow}.
	
		\subsubsection{fyring}
		Dette er det letteste tilfellet. Ved fyring har vi allerede beregna eit eksakt fyringstidspunkt.
		Fyringstidspunktet eg også definert som start--tid for neste 'time window', og  vi trenger bare lagre dette tidspunktet.

		Første vi må gjøre er å definere \emph{double dEstimatedTaskTime} istadenfor \emph{unsigned long ulEstimatedTaskTime}.
		I \emph{K\_auron::doTask()} lagrer vi auronets \emph{dEstimatedTaskTime} til \emph{dStartOfTimeWindow}.
		TRUR DETTE VIL GI OSS PERFEKT OPPLADING AV NEURONET!
	
% TODO 1) Lagre estimert tidspunkt i en double-variabel istadenfor en unsigned long: double dEstimatedTaskTime 			ikkje lenger ulEstimatedTaskTime.. 														TODO (1)
% TODO 2) Lagre tidspunkt for start av 'time window' i en double istadenfor unsigned long: 	dStartOfTimeWindow 			ikkje lenger ulStartOfTimewindow.. 														TODO (2)
%		 	2.1) Trenger å endre på getCalculateDepol() slik at denne kan kalkulere depol for eit float-tidspunkt. Argument!
% 				Har innført getCalculateDepol(double). KOMMENTERT UT! Går for løyringa under: innfører variabel 		dNextStartOfTimeWindow
% TODO  	2.2) Trenger å ha inn tidspunkt til doCalculations(): kanskje bedre å innføre variable dNextStartOfTimeWindow? (gjelder også den over..)
% 			2.3) Har delt opp doCalculation() i [beregn ny kappa] og [estimer periode]. Estimering av periode skjer no i funksjonen protected inline void K_auron::estimatePeriod();
		\subsubsection{input til neuronet}
			Når neuronet får endret input, vil tidsspunkt for endring være viktig for å beholde eksakt løysing.
			(I starten tenker eg å gjøre dette lett, og sette $t_0$ til enten starten av iterasjonen eller midten av iteraasjonen. (trur dette gir samme resultat i lange løp). 
				Etterkvart må eg utvide dette til å være gitt av input. Input kan for eksempel ha ved en "phase", som gir når den er gyldig. 
				Dersom vi lar postsyn. $t_0$ være gitt av overføringens $t_0$ skalert med inflytelsen til overføringen, kan vi få til dette målet. Jobb meir med dette seinare).

		Foreløpig sjekker vi en sensor node(som sampler eit input, og ikkje har noko neuralt input).
		Vi kan da definere at sampling skjer ved starten av kvar tidsiterasjon.
		Dermed kan vi definere at 'time window' (float-tidspunkt) er det samme som tidsiterasjonens nummer (unsigned long-tidspunkt - sample nummer).

% TODO 3) Lagre starten av 'time window', initialisert av en endret senset variabel, $\kappa$, ved å bare typekonvertere tidsiterasjonen: 			dStartOfTimeWindow = (double)ulTime; //(sett den til "no") 	TODO (3)

	\subsection{Normalisert tid: mulighet for å endre tidsoppløysinga}
	For å få samme transiente forløp for simuleringer med ulik tidsoppløsning, må vi ikkje bare referere til tid som tidsstegnummer. 
	Vi må da skalere på basis av lengden på computational time step, $\Delta t$.
	Dette kan vi gjøre ved å skalere med $\alpha$ (bare lekkasjen [OG OGSÅ MENGDEN INPUT] som er avhengig av lengden på eit tidssteg.
	Finn ut om eg kan løse dette bare ved å endre $\alpha$!
	%- Skaler både den nedskrevene tida (i log-filene) og tida som sendes inn i sensor-funksjon. Det viktigasete er det siste, men dei henger sammen..

	%TODO Plasser all kode som har med å skriv til logg i mdl.funk. writeDepolToLog() -også for SANN: TODO
	%TODO 	Så innfør skalering i denne funksjonen. (da blir alt slikt samla på samme sted i koden (KANN og SANN writeDepolToLog() kan ligge rett ved kvarandre!


%%%%%%%% FRA FDP: %%%%%%%%%%%%

\section{Tid}

For å ikkje måtte oppdatere alle objekta i neuralnettet kvar tidsiterasjon, har eg laga konseptet: flytende tid.

Grunnen til å ha tid i det heile tatt er:
\begin{itemize}
	\item at en del aspekter for objekta er tidsvarinte (funksjon av tid). Bl.a. 'refraction time' og lekkasje.
	\item for å få rett kausalitet i det simulerte sytemet. --at  $b_2$ skjer ETTER $b_1$, men samtidig med $a$ om $a$ og $b_2$ skal skje samtidig. Får (simulert) samtidighet.
\end{itemize}

Dette med simulert samtidighet er også eit poeng. Vidare skal vi også få inn 'refraction time' for neurona. %XXX

For å få til dette med tid, uten at alle objekta må skjekkes kva tidsiterasjon, vil eg bruke konseptet flytende tid. For å beskrive dette, må eg først beskrive systemets 'scheduler'.

\subsection{'Scheduler'}
For å få til kausalitet (at $a_2$ skal skje etter $a_1$, men ikkje etter $b$) har eg laga en 'scheduler'. 

Se for deg ei lenka liste som inneholder alle jobbane, pNesteJobbArbeidsKoe. En (alltid nøyaktig en) av jobbane er en spesiell tybe jobb, kalla tid.doTask();. I tillegg har eg en (globalt tilgjengelig) variabel kalla ulTidsiterasjoner.

Alle jobbane har muligheten for å skape nye jobber, som i såfall legges på på slutten av pNesteJobbArbeidsKoe.

Når \emph{tid.doTask()} kalles, vil denne gjøre de aktiviteter som skal gjøres i alle tidsiterasjonane og iterere tid ( ulTidsiterasjoner++ ).
I tillegg skal denne legge på en \emph{[this]} peiker på slutten av arbeidslista \emph{pNesteJobbArbeidsKoe}.

Dette vil skape en lenka liste variant av at det er to arbeidskøer som altererer, med tidsiterator kvar gang ei av de er tom og vi skifter til andre arbeidslista.



\subsection{Klassene}
Alle som skal schedules skal arve fra class tidInterface. Alle klassene som skal ha timing (causalitet og tidsvariante funksjoner) skal arve fra tidInterface. 
Dette er ei interface-klasse med \emph{virtal void doTask() =0;}

I tillegg til alle klassene som har elementer som er funksjoner av tid (for eksempel lekkasjen i 'leaky integrator'), skal en spesialklasse \emph{class tid} arver fra tidInterface.

\subsubsection{tid::pNesteJobbArbeidsKoe}
\emph{tid::pNesteJobbArbeidsKoe} inneholder alle jobber som skal gjøres. Dersom denne er tom, slutter programmet å gjøre ting (i tillegg til at tid slutter --også \emph{tid} er eit element i arbeidslista).

Dersom en jobb skaper nye jobber, legges desse til på slutten av lista.

Ei klasse er heilt spesiell når det kommer til tid: i tillegg til å inneholde \emph{static list pNesteJobbArbeidsKoe}, fungerer tidsSkilleElement som eit skilleElement mellom to tidsiterasjoner. 
Når \emph{tid.doTask()} kalles, skal alt som har med tidsprogresjonen gjøres. I tillegg tenker eg å putte funksjoner som går som bare en funksjon av tid inn her (alt som skal skje i kvar tidsiterasjon)..

Men det viktigaste \emph{tid.doTask()} gjør, er å iterere tid.
\emph{tid} har dermed ansvaret for å skille mellom to tidsiterasjoner.

\subsection{implementasjon}
Som sagt: alle klasser som har ei oppgave som skal gjøres (i den simulerte tida) puttes inn i \emph{pNesteJobbArbeidsKoe}. 

Selve schedulinga skjer i 
\begin{equation}
	\text{void* schedulerFunskjon(void*)}
\end{equation}
Grunnen til formatet på funksjonen er muligheten for å starte tråder (konvensjoner i pthread--library).

Denne funksjonen kaller pNesteJobbArbeidsKoe.front()->doTask(). Når pNesteJobbArbeidsKoe er av type [\emph{class tid}] vil tid itereres.

I tillegg skal alt anna som skal skje kvar einaste iterasjon plasseres her. Eksempel kan være at nye synapser blir lagt i ei kø, og schedulerFunskjon går gjennom denne køa kvar gang den kalles. 
Events kan kanskje også plasseres her?

Eller eg kan gjøre desse andre tinga med egne tråder?



\section{Klassestruktur}
Klassene er bygd opp som i uml-klassediagrammet i loggboka. I tillegg tenker eg å ha interface-klasser for auron og synapse

Forskjellen mellom SANN og KANN er måten aktiviteten propagerer. Dersom eg da har eit aktivitetsObj med relevante funskjoner (som sendSignal(), getDepol(), osv), så kan synapse, dendrite og auron være det samme for ANN og SANN (bare med ulikt aktivitetsObj (nedarva fra i\_aktivitetsObj, som er interfaceklasse for begge aktivitetsObj--typer).


\subsection{interfaceklasser}
For å få det så generellt som mulig, lager eg ei interface-klasse (f.eks. i\_auron) som er generalisering av spiking auron og $\kappa$ auron. Samme for alle auron--element (synapse,auron(soma),axon, dendrite).

Interfaceklassen inneholder alle funksjoner og variabler som er generellt for begge underklassene, dvs alle de funksjoner og variabler som får ANN til å gå.





\subsection{peikere}
Siden eg heile tida opererer med peikere, og lager alle objekt i det frie lageret, trenger eg å eksplisitt kalle destruktor.
Siden synapse kvar synapse er tilknytta bare en presynatisk og en postsynaptisk node, begynner eg destrueringa her. Når f.eks. eit axon skal destrueres går eg gjennom alle pUtSynapser--element, og destruerer.
\begin{lstlisting}
while( !pUtSynapser.empty() )
	delete ( *pUtSynapser.begin() )
\end{lstlisting}

\include{SANN}
\include{KANN}

\include{KANN-SANN_sammenligning.tex}

\newpage
\section{Dendrite}
Når det kommer til dendritt, tenker eg å holde muligheten open for å utvide. Foreløpig lar eg kvart auron ha bare en dendritt: [\emph{\small{dendrite dendritt\_input;}}].
For bedre simulering kan denne endres til [\emph{\small{std::list$<$dendrite*$>$ alleDentritter;}}].





\section{Synapse}

\subsection{SANN}
Synaptisk vekt skal være unsigned i staden for int!

Ved kvar overføring skal synaptisk vekt legges til eller trekkes fra i postsyn. depol. (avhengig av synapse::bInhibEffekt). 

%Veit ikkje om neste er gyldig lenger. Trur kanskje det. Gidd ikkje sjekke no.
%Først sjekkes det om kor lenge det er siden syn.overføring.  (skjer i synapse::transmission() )
%\begin{lstlisting}
%if( (unsigned uTidSiden = ulTid - ulForrigeOppdatering) != 0) % eller kanskje heller: if(ulTid != ulForrigeOppdatering), og kjøre sammenligninga direkte i pow-funk i gjennomførLekkasje (unngår argumentkopiering..)
%	gjennomførLekkasje(uTidSiden);
%\end{lstlisting}
%
%og gjennomførLekkasje() :
%\begin{lstlisting}
%void gjennomførLekkasje(uTidSiden_arg){
%	depol = lekkasjeFaktor ^ uTidSiden_arg		
%}
%\end{lstlisting}


\subsection{KANN}
KANN--synapsene skal regne ut synapsens innvirking på postsyn. auron. Dette er gitt av presyn. periode og synaptisk vekt. (her kan eg kanskje seinare også legge inn en temp-syn.vekt variabel for å lage short-term syn.p.)

Det er ``synaptisk overføring'' kvar gang presyn. node endrer $\kappa$. 
For at det skal være effektivt i postsyn. dendrite, regner synapsen ut \emph{endring} i synaptisk overføring og sender denne videre. 
For å gjøre dette, trenger synapsen en variabel som holder styr på $\text{[presyn. periode]}^{-1}$, og når vi får synaptisk overføring regner synapsen ut differansen (deriverte) av endinga, og sender dette som argument til dendrite.

\begin{lstlisting}
void synapse::overfoering(){
	unsigned nyPeriode_temp = pMeldemAvAuron->periode; 
	//regnes ut ved A.P. i auronet.
	
	pPostsyn_Dendrite->nyttSignal( synOverfoeringDerivat );  
	//synOverfoeringDerivat er medlemsvariabel (sjaa under)
	
	synOverfoeringDerivat = periode - nyPeriode_temp;
	periode = nyPeriode_temp;
}
\end{lstlisting}

Eller noke tilsvarende.. 

Siden K\_dendrite::nyttSignal( derivat\_arg ) tar derivat som argument, kan vi i dendrite kjøre kallet $K_i += \text{[derivat\_arg]}$. 
For å unngå integral--avvik bør vi kjøre en regelmessig sjekk som rekalkulerer $\kappa_i$ for neuron $i$.
Dette kan gjøres ved å summere alle inn--synapsenes innvirkning (produkt av medlemsvariablane $\text{preSynPeriode]}^{-1}$ og $W_{ij}$.

Kor ofte denne bør kjøres bør enten finnes eksperimentellt, eller gjøres dynamisk (type: dersom integral--avviket er for stort så minskes den regelmessige perioden..).

\newpage





\bibliography{bibliografi}
%\bibliographystyle{abbrvnat}
\bibliographystyle{plain}
\end{document}

